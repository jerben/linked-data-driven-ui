\section{Introduction}\label{introduction}

This thesis explores the generation of user interfaces (UIs) for application programming interfaces (APIs) that fulfill certain criteria in the context of the World Wide Web.

\subsection{Context}\label{context}
The creation of a software product today usually involves some kind of UI implementation, which the customer will use to interact with the product. Often administrators should be able to do some backoffice tasks without having access to the persistence layer of the product, so a second admin UI is required. Systems today often implement HTTP APIs using JSON. The UIs are separate projects that contain clients which consume those APIs.

TODO mention linked data and task based computing

\subsection{Problem}\label{problem}
The contemporary way of implementing those UIs involves a lot of manual work. The JSON data formats and schemas being used are often agreed on on a case to case basis. While humans can often infer the meaning of a piece of JSON data, it is not trivial for machines to do so. Specific knowledge about the JSON data is needed for a machine to understand the meaning of a JSON blob. That specific knowledge is \textbf{context}, which is usually encoded in the client/UI implementation during a manual development process.

TODO not leverageing web architecture properly -> linked data, what is task based computing?

\subsection{Strategy}\label{strategy}
The main goal of this thesis is to reduce the amount of manual work that goes into UI development. Firstly, I will give an overview of existing technologies and tools and highlight how they fail to address the problem. Then I will introduce the building blocks that could be used to generate UIs. Lastly, I define a set of three use cases for UI generation. I am using the building blocks to implement a UI generation framework that tries to solve those use cases.

TODO design principles: play nice with existing APIs, extensibility, sane defaults
