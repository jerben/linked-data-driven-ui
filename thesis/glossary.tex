\newglossaryentry{hateoas}
{
    name={HATOAS},
    description={test}
}

\newglossaryentry{hypermedia}
{
    name={Hypermedia},
    description={Non-linear medium of information containing plain text, audio, video and hyperlinks}
}

\newglossaryentry{crud}
{
    name={CRUD},
    description={Create, Read, Update and Delete: Four basic operations of persistence storage}
}

\newglossaryentry{semanticweb}
{
    name={Semantic Web},
    description={An extension of the World Wide Web in form of standards and tools}
}

\newglossaryentry{sparql}
{
    name={SPARQL},
    description={SPARQL Protocol and \gls{rdf} Query Language: Used to fetch and manipulate \gls{rdf} data}
}

\newglossaryentry{quadstore}
{
    name={quad store},
    description={\gls{rdf} store where the data is stored as triples \textit{subject-predicate-object} and name}
}

\newglossaryentry{rdf}
{
    name={RDF},
    description={Resource Description Framework: A set of specifications, used as data serialization format}
}

\newglossaryentry{dsl}
{
    name={DSL},
    description={Domain Specific Language: A restricted language used to express problems in a specific domain as opposed to general purpose languages}
}

\newglossaryentry{cognitive load}
{
    name={cognitive load},
    description={The level of mental effort it takes a developer to write software}
}

\newglossaryentry{pagination}
{
    name={pagination},
    description={The process of slicing a collection into smaller adjacent subsets that can be browsed page by page}
}

\newglossaryentry{linkeddata}
{
    name={linked data},
    description={Structured data which is interlinked with other data with the aim to provide semantics}
}

\newglossaryentry{console}
{
    name={console},
    description={As opposed to a web client, a console holds no state and no business logic}
}

\newglossaryentry{uri}
{
    name={URI},
    description={Uniform Resource Identifier refers to a string that identifies a resource}
}

\newglossaryentry{url}
{
    name={URL},
    description={Uniform Resource Locator refers to a subset of \gls{uri} that provides a description of the access path, additionally to uniquely identify the resource}
}

\newglossaryentry{backend}
{
    name={back-end},
    description={The server providing an API that is used by the \gls{frontend}}
}

\newglossaryentry{frontend}
{
    name={front-end},
    description={The client containing a UI that runs in the browser, as mobile application or as desktop program. It communicates with a server, referred to as \gls{backend}}
}

\newglossaryentry{ux}
{
    name={UX},
    description={The user experience (UX) describes the way a user experiences a UI}
}
