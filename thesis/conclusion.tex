\section{Conclusion}\label{conclusion}

\subsection{Conclusion of the UI framework}
We conclude by analyzing the implementation of the design goals and the contribution of the UI framework towards reducing complexity and assess whether the use can reduce costs in UI development.

\end{table}
\begin{table}[!htb]
  \begin{center}
    \begin{tabular}{|l|l|l|}
      \hline
      \textbf{ID} & \textbf{Title} & \textbf{Degree to which goal is met} \\
      \hline
      D1 & Play nicely with others & 3 \\
      \hline
      D2 & Straightforward upgrade path & 2.5 \\
      \hline
      D3 & Customizability & 3 \\
      \hline
      D4 & Developer ergonomics & 2 \\
      \hline
    \end{tabular}
    \caption{Overview of the design goals and their score indicating the degree to which they are met.}
  \end{center}
\end{table}


\begin{table}[!htb]
  \begin{center}
    \begin{tabular}{|l|l|l|l|}
      \hline
      \textbf{ID} & \textbf{Title} & \textbf{Type} & \textbf{Change in complexity} \\
      \hline
      C1 & Complexity caused by Invisibility & Essential & 1 \\
      \hline
      C2 & Complexity caused by Conformity & Essential & 0 \\
      \hline
      C3 & Complexity caused by Changeability & Essential & 3 \\
      \hline
      C4 & Complexity caused by Complexity & Essential & unknown \\
      \hline
      C5 & Complexity caused by State & Accidental & 3 \\
      \hline
      C5 & Complexity caused by Control & Accidental & 2 \\
      \hline
    \end{tabular}
    \caption{Overview of the types of complexities in software development and how the UI framework affects them. Positive scores indicate a reduction of complexity and negative scores indicate an increase.}
    \label{tab:summarycomplexity}
  \end{center}
\end{table}

\subsection{Review}
The design goals are mostly met. The UI framework plays nicely with existing HTTP and JSON tooling due to the use of JSON-LD. The upgrade path of an existing system with a JSON HTTP API and a client to Hydra and the UI framework is straightforward. The UI framework is highly customizable. It is possible to do traditional web UI development but the framework guides developers to create a \gls{console} rather than the usual app. The development workflow using the default renderers is ergonomic. \\
The migration of complex POST requests to Hydra might lead to non-idiomatic use of Hydra operations. There is also a non-trivial learning curve when it comes to Hydra - both on the consumer and producer side. This is especially the case if the interaction with the UI is not trivial.

The complexity is reduced compared to contemporary UI development as described in section \ref{problem}. This is mostly due to eradicating application state in the client and allowing the API to evolve with very little friction. In general, the cognitive load is reduced by providing the developer with this framework of mapping \gls{linkeddata} types to components.

\subsection{Free lunch}
According to table \ref{tab:summarycomplexity} there is a positive net total of complexity that vanished. Even after considering the learning curve for \gls{linkeddata}, JSON-LD and Hydra - the costs of UI and client development are drastically reduced. \\
According to the definition of Essential Complexity by Brooks it is not possible to get rid it. C0, C1, C2 and C3 are Essential Complexity and there is a reduction in total. \\
Our analysis considered the point of view of UI developers - not of developers implementing the Hydra API. Complexity is reduced in the client but it increases on the server.

Setup of Hydra on the server side brings nonnegligible overhead. Serialization of the data model and routing of operations have to be considered. In more complex use cases, it is not sufficient to just serialize the domain model to JSON-LD. Since the client should not contain business logic it is not able to denormalize data on its own. The server has to consider \textbf{how the data is consumed and it has to denormalize it accordingly}. The Hydra API implementation contains knowledge about how the end user interacts with the system because the client is merely a \gls{console} that holds no state. \\
Not often are developers working on the \gls{backend} the same developers that design the UX. UX designers and UX developers are required to contribute on HTTP API level. In contemporary UI development the HTTP API is often the boundary between the team that focuses on the user and how the user is interacting with the system and the team that focuses on data modeling. With the approach of the UI framework and Hydra this line becomes blurred.

Nevertheless, the complexity lives in one place. It is not obvious why UX knowledge can not exist in the server on a generic level. The clients that are developed using the UI framework implement the UX by providing knowledge about graphics, design, styling and usability. The use of Hydra and the UI framework \textbf{allow the complexity to be contained and managed on the server}.

Adoption of Hydra, \gls{linkeddata} and the UI framework is a total net gain.

\subsection{Future work}
The HTTP API implementations of both use cases are hard coded. This thesis provides a proof of concept for a Hydra \gls{console}. Generic generation of an API that conforms to Hydra is not in the scope of this thesis. It is obvious that a tool that creates a Hydra API given a data model or an existing HTTP API together with the UI framework could make a full-stack \gls{linkeddata}-driven application framework. \\
A possible approach is to implement Hydra serialization for an existing platform. Rails makes heavy use of Convention over Configuration which allows its plugins (gems) to make a lot of assumption. A Rails gem has direct access to the data model. This could be used to expose operations, generate documentation and serialize models to JSON-LD. \\
Building a Hydra plugin on top of an established platform fits the central design goal and philosophy to \textbf{play nicely with others}.
